\section{引言}
\subsection{研究工作的背景及意义}
随着科技与信息技术的发展,感知不再是雷达的专属功能,通信信号与雷达信号之间的界限也逐渐变得模糊。从蜂窝通信的角度看,通信感知一体化被认为是下一代移动通信(6G)中的核心功能,要求通信设备兼具通信与感知的能力\cite{ISAC}。从室内无线局域网(Wi-Fi)通信的角度看,电子与电气工程师协会(IEEE: Institute of Electrical and Electronics Engineers)于2020年9月成立了专门针对Wi-Fi感知的工作小组IEEE 802.11bf (Task Group bf)\cite{TGbf_PAR}。IEEE 802.11bf的一项提案``WiFi Sensing Uses Cases''\cite{TGbf_Use_Case}指出:基于Wi-Fi信号对周围环境进行重构是Wi-Fi感知的一个重要应用方向,通过成像的方式则是环境重构的一种途径。


目前,传统的成像方案主要借助于雷达信号,或在现有通信信号中引入雷达波形,这极大了加大了设计与实现的复杂度,而基于原本的Wi-Fi波形(不借助任何的雷达信号)通过Wi-Fi感知的方式进行成像,则为我们提高了一种全新的、低复杂度的新思路。这也是本文研究最大的动机:通过Wi-Fi近场成像的方式对室内环境进行重构。
\subsection{国内外研究现状}
目前,国内外对于成像的研究主要集中于室外的远场成像,并已经有了许多成熟的方案,如基于合成孔径雷达(SAR: Synthetic Aperture Radar)的远场成像\cite{indoorSAR},以及基于全球卫星导航系统(GNSS: Global Navigation Satellite System, GNSS)以及激光雷达或计算机视觉的SLAM(Simultaneous Localization And Mapping)等\cite{NavigationSAR}。但是,这些室外成像的解决方案并不可以直接地套用到室内近场成像中,例如SAR成像往往需要引入线性调频信号(FMCW: Frequency Modulated Continuous Wave )等雷达信号波形而增大设计地复杂度。而基于GNSS的SLAM则因卫星无法在室内覆盖以及激光雷达、计算机视觉等带来的其他问题更加地无法在室内适用。


另一个方面,基于Wi-Fi的感知技术,目前的绝大多数工作主要集中在基于Wi-Fi信号的接收信号强度(RSSI: Received Signal Strength Indicator)或信道状态信息(CSI)做环境或人体的检测,识别,和估计等\cite{ma2019wifi},但是对于空间感知方面还没有太多的深入研究。因此,通过Wi-Fi感知的方式进行成像具有巨大的研究潜力。

此外,一些基于毫米波的成像工作也值得探索。例如,一些研究基于60GHz毫米波通信的接收多径具有稀疏性的性质,利用$l_1$正则化最小二乘法估计接收机每一条多径的传输距离(主要体现为多径中的参数TOF: Time of Flight)与入射角度(主要体现为多径中的参数AOA: Angle of Arrival)并构建所谓的``距离·角度图''。
另外,同样基于毫米波的稀疏性,文章\cite{yang2022hybrid}提出了一种基于因子图上的信息传输(message passing)算法进行贝叶斯估计的radio-SLAM方法。这些工作都为毫米波成像提供了可能途径\cite{barneto2021millimeter}。

\subsection{主要研究内容}
本文提出了一种基于Wi-Fi被动感知的室内近场匹配滤波成像的方法,可以有效地对信源,反射物,散射点等进行检测,定位,和成像。另外,为了应对实际通信系统中频偏带来的影响,本文还设计了基于CSI共轭相乘消除频偏以及结合平行因子(PARAFAC)分析消除频偏的成像方案。对于本技术的潜在应用场景,本文也设计了一个简单的分布式联合成像方案辅以验证。对于以上方案,本文均做出了相应的仿真以验证。为了验证方案的可行性,本文还通过数控轨道搭载接收天线移动模拟大规模线性接收阵列做出简单的实测成像验证。

\subsection{论文结构组织}
论文的主要结构如下:
\begin{itemize}
    \item 第一章是论文的引言部分,主要介绍了论文的研究背景、工作意义、国内外研究现状、主要研究内容,以及文章的结构组织。
    \item 第二章介绍了基于Wi-Fi的近场成像技术的系统模型以及对理想无频偏影响系统的仿真。
    \item 第三章介绍了CSI频偏模型,以及基于CSI共轭相乘的频偏消除成像和结合共轭相乘和PARAFAC算法的频偏消除成像。
    \item 第四章介绍了该技术的潜在应用场景:分布式成像,并给出了一个简单的范例和仿真。
    \item 第五章给出了一个简单的实测,通过数控轨道搭载接收天线移动模拟大规模线性接收阵列进行室内成像验证。
    \item 第六章是本文的总结部分。
    \item 剩下的章节是本文的参考文献,附录,以及致谢部分。
\end{itemize}