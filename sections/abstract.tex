% !Mode:: "TeX:UTF-8"
% !TEX program  = xelatex
\begin{中文摘要}{通信感知一体化,Wi-Fi被动感知,近场匹配滤波,信道状态信息,平行因子}
作为6G的一项新的核心功能,通信感知一体化(ISAC: Integrated Sensing And Communication)越发引起人们的关注,利用现有通信手段对空间环境进行感知、重构是其中一个重要的议题。本文提出了一种基于Wi-Fi被动感知的室内近场成像技术,通过阵列信号的相干、近场匹配滤波来实现室内对信源和反射物的探测、定位与成像。为了解决实际通信系统中频偏带来的影响,本文还提出了基于信道状态信息(CSI: Channel State Information)共轭相乘以消除频偏,以及结合平行因子(PARAFAC: PARAllel FACtor)分析技术以消除频偏的成像方案。另外,对于本技术的潜在应用场景,本文提出了一种室内搭载多个分布式阵列联合成像的场景,并对其进行了仿真。为了验证成像的可行性,本文还对单阵列室内成像进行了简单实测。
\end{中文摘要}

\begin{英文摘要}{ISAC, Wi-Fi Passive Sensing, Near-Field Matched Filtering, CSI, PARAFAC}
As one of main new features of 6G, Integrated Sensing And Communication (ISAC) increasingly draws people's attentions, in which sensing and reconstructing surrounding environments making use of existing communication techniques is one of the most important topic. This article proposes an indoor near field imaging method, which implements the detection, localization, and imaging of indoor sources and reflection objects, based on Wi-Fi Passive Sensing and processing array signals via coherent superposition and Near-Field Matched Filtering. In order to combat with the impacts brought by frequency offset, this article also presents imaging schemes that correcting frequency offset via Channel State Information (CSI) conjugate multiplication or combining CSI conjugate multiplication with Parallel Factor (PARAFAC) analysis. In addition, for the potential application scenarios, an indoor scheme using multiple distributed received sub-arrays to process imaging is discussed and simulated. To verify the feasibility of imaging, a simple indoor imaging experiment using one received array is also given.
\end{英文摘要}
